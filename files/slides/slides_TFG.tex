\documentclass[aspectratio=43]{beamer}
% \documentclass[aspectratio=169]{beamer}

% Title --------------------------------------------
\title{Información TFG\\Ciencias Políticas / Estudios Internacionales}
\author{Francisco Villamil\\(\texttt{francisco.villamil@uc3m.es})\\Coordinador de TFG de CCPP y EEII}
\date{Curso 2023--2024}

%%% NOTE -- CHECK THIS: https://github.com/paulgp/beamer-tips


%%% Building heavily on https://github.com/kylebutts/templates

% xcolor, define them
\usepackage{xcolor}

% TEXT COLORS
\definecolor{red}{HTML}{9a2515}
\definecolor{yellow}{HTML}{EBC944}
\definecolor{asher}{HTML}{555F61}
\definecolor{jet}{HTML}{131516}

% THEME COLORS
\definecolor{accent}{HTML}{107895}
\definecolor{accent2}{HTML}{9a2515}

% Color commands
\newcommand\red[1]{{\color{red}#1}}
\newcommand\yellow[1]{{\color{yellow}#1}}
\newcommand\asher[1]{{\color{asher}#1}}

\newcommand\BGred[1]{{\colorbox{red!80!white}{#1}}}
\newcommand\BGyellow[1]{{\colorbox{yellow!80!white}{#1}}}
\newcommand\BGasher[1]{{\colorbox{asher!80!white}{#1}}}

\renewcommand<>{\BGyellow}[1]{\only#2{\beameroriginal{\BGyellow}}{#1}}

% Appendix numbering
\usepackage{appendixnumberbeamer}

% Beamer Options -------------------------------------

% Background
\setbeamercolor{background canvas}{bg = white}

% Change text margins
\setbeamersize{text margin left = 25pt, text margin right = 15pt}

% \alert
\setbeamercolor{alerted text}{fg = accent2}

% Frame title
\setbeamercolor{frametitle}{bg = white, fg = jet}
\setbeamercolor{framesubtitle}{bg = white, fg = accent}
\setbeamerfont{framesubtitle}{size = \small, shape = \itshape}

% Block
\setbeamercolor{block title}{fg = white, bg = accent2}
\setbeamercolor{block body}{fg = jet, bg = jet!10!white}

% Title page
\setbeamercolor{title}{fg = jet}
\setbeamercolor{subtitle}{fg = accent}

%% Custom \maketitle and \titlepage
\setbeamertemplate{title page}
{
    \begin{centering}
      % \vspace{20mm}
      {\Large \usebeamerfont{title}\usebeamercolor[fg]{title}\inserttitle}\\ \vskip0.25em%
      \ifx\insertsubtitle\@empty%
      \else%
        {\usebeamerfont{subtitle}\usebeamercolor[fg]{subtitle}\insertsubtitle\par}%
      \fi%
      {\vspace{10mm}\insertauthor}\\
      \ifx\insertinstitute\@empty%
      \else%
        {\vspace{5mm}\color{asher}\scriptsize{\insertinstitute}}\\\vspace{5mm}
      \fi%
      {\color{asher}\small{\insertdate}}\\
    \end{centering}
}

% Table of Contents
\setbeamercolor{section in toc}{fg = accent!70!jet}
\setbeamercolor{subsection in toc}{fg = jet}

% Button
\setbeamercolor{button}{bg = accent}

% Remove navigation symbols
\setbeamertemplate{navigation symbols}{}

% Table and Figure captions
\setbeamercolor{caption}{fg=jet!70!white}
\setbeamercolor{caption name}{fg=jet}
\setbeamerfont{caption name}{shape = \itshape}

% Put slide number / total slides at the bottom right
\makeatother
\makeatletter
\setbeamertemplate{footline} %{\hfill\insertframenumber/\inserttotalframenumber}
{%
  \leavevmode%
  \hbox{
  \begin{beamercolorbox}[wd=\paperwidth,ht=2.5ex,dp=1.125ex,leftskip=.3cm,rightskip=.3cm plus1fil]{footlinecolor}%
    \color{asher}{\hfill\insertframenumber/\inserttotalframenumber}
  \end{beamercolorbox}}%
  \vskip0pt%
}
\makeatother
\makeatletter

% Bullet points

%% Fix left-margins
\settowidth{\leftmargini}{\usebeamertemplate{itemize item}}
\addtolength{\leftmargini}{\labelsep}

%% enumerate item color
\setbeamercolor{enumerate item}{fg = accent}
\setbeamerfont{enumerate item}{size = \small}
\setbeamertemplate{enumerate item}{\insertenumlabel.}

%% itemize
\setbeamercolor{itemize item}{fg = accent!70!white}
\setbeamerfont{itemize item}{size = \small}
\setbeamertemplate{itemize item}[circle]
\setlength{\itemsep}{0pt plus 6pt}

%% right arrow for subitems
\setbeamercolor{itemize subitem}{fg = accent!60!white}
\setbeamerfont{itemize subitem}{size = \small}
\setbeamertemplate{itemize subitem}{$\rightarrow$}

\setbeamertemplate{itemize subsubitem}[square]
\setbeamercolor{itemize subsubitem}{fg = jet}
\setbeamerfont{itemize subsubitem}{size = \small}

% References

%% Bibliography Font, roughly matching aea
\setbeamerfont{bibliography item}{size = \footnotesize}
\setbeamerfont{bibliography entry author}{size = \footnotesize, series = \bfseries}
\setbeamerfont{bibliography entry title}{size = \footnotesize}
\setbeamerfont{bibliography entry location}{size = \footnotesize, shape = \itshape}
\setbeamerfont{bibliography entry note}{size = \footnotesize}

\setbeamercolor{bibliography item}{fg = jet}
\setbeamercolor{bibliography entry author}{fg = accent!60!jet}
\setbeamercolor{bibliography entry title}{fg = jet}
\setbeamercolor{bibliography entry location}{fg = jet}
\setbeamercolor{bibliography entry note}{fg = jet}

%% Remove bibliography symbol in slides
\setbeamertemplate{bibliography item}{}





% Links ----------------------------------------------

\usepackage{hyperref}
\hypersetup{
  colorlinks = true,
  linkcolor = accent2,
  filecolor = accent2,
  urlcolor = accent2,
  citecolor = accent2,
}


% Line spacing --------------------------------------
\usepackage{setspace}
\setstretch{1.2}


% \begin{columns} -----------------------------------
\usepackage{multicol}


% % Fonts ---------------------------------------------
% % Beamer Option to use custom fonts
% \usefonttheme{professionalfonts}
%
% % \usepackage[utopia, smallerops, varg]{newtxmath}
% % \usepackage{utopia}
% \usepackage[sfdefault,light]{roboto}
%
% % Small adjustments to text kerning
% \usepackage{microtype}



% Remove annoying over-full box warnings -----------
\vfuzz2pt
\hfuzz2pt


% Table of Contents with Sections
\setbeamerfont{myTOC}{series=\bfseries, size=\Large}
\AtBeginSection[]{
        \frame{
            \frametitle{Roadmap}
            \tableofcontents[current]
        }
    }


% References ----------------------------------------
\usepackage[
    citestyle= authoryear,
    style = authoryear,
    natbib = true,
    backend = biber
]{biblatex}

% Smaller font-size for references
\renewcommand*{\bibfont}{\small}

% Remove "In:"
\renewbibmacro{in:}{}

% Color citations for slides
\newenvironment{citecolor}
    {\footnotesize\begin{color}{accent2}}
    {\end{color}}

\newcommand{\citetcolor}[1]{{\footnotesize\textcolor{asher}{\citet{#1}}}}
\newcommand{\citepcolor}[1]{{\footnotesize\textcolor{asher}{\citep{#1}}}}

% Tables -------------------------------------------
% Tables too big
% \begin{adjustbox}{width = 1.2\textwidth, center}
\usepackage{adjustbox}
\usepackage{array}
\usepackage{threeparttable, booktabs, adjustbox}

% Fix \input with tables
% \input fails when \\ is at end of external .tex file

\makeatletter
\let\input\@@input
\makeatother

% Tables too narrow
% \begin{tabularx}{\linewidth}{cols}
% col-types: X - center, L - left, R -right
% Relative scale: >{\hsize=.8\hsize}X/L/R
\usepackage{tabularx}
\newcolumntype{L}{>{\raggedright\arraybackslash}X}
\newcolumntype{R}{>{\raggedleft\arraybackslash}X}
\newcolumntype{C}{>{\centering\arraybackslash}X}

% Figures

% \imageframe{img_name} -----------------------------
% from https://github.com/mattjetwell/cousteau
\newcommand{\imageframe}[1]{%
    \begin{frame}[plain]
        \begin{tikzpicture}[remember picture, overlay]
            \node[at = (current page.center), xshift = 0cm] (cover) {%
                \includegraphics[keepaspectratio, width=\paperwidth, height=\paperheight]{#1}
            };
        \end{tikzpicture}
    \end{frame}%
}

% subfigures
\usepackage{subfigure}


% Highlight slide -----------------------------------
% \begin{transitionframe} Text \end{transitionframe}
% from paulgp's beamer tips
\newenvironment{transitionframe}{
    \setbeamercolor{background canvas}{bg=accent!60!black}
    \begin{frame}\color{accent!10!white}\LARGE\centering
}{
    \end{frame}
}


% Table Highlighting --------------------------------
% Create top-left and bottom-right markets in tabular cells with a unique matching id and these commands will outline those cells
\usepackage[beamer,customcolors]{hf-tikz}
\usetikzlibrary{calc}
\usetikzlibrary{fit,shapes.misc}

% To set the hypothesis highlighting boxes red.
\newcommand\marktopleft[1]{%
    \tikz[overlay,remember picture]
        \node (marker-#1-a) at (0,1.5ex) {};%
}
\newcommand\markbottomright[1]{%
    \tikz[overlay,remember picture]
        \node (marker-#1-b) at (0,0) {};%
    \tikz[accent!80!jet, ultra thick, overlay, remember picture, inner sep=4pt]
        \node[draw, rectangle, fit=(marker-#1-a.center) (marker-#1-b.center)] {};%
}


\usepackage[T1]{fontenc}
\usepackage[utf8]{inputenc}
\usepackage[spanish]{babel}

\begin{document}
% ====================================================

% ----------------------------------------------------
\begin{frame}
  \titlepage
\end{frame}
% ----------------------------------------------------

% ----------------------------------------------------
\begin{frame}
\frametitle{Básicos}
\centering

\begin{itemize}
  \item Información Departamento de Ciencias Sociales
  \item[] \BGyellow{\url{https://ccss-uc3m.github.io/TFG/}}
  \item[]
  \item Información general Facultad de Ciencias Sociales y Jurídicas
  \item[] {\footnotesize \url{https://www.uc3m.es/ss/Satellite/SecretariaVirtual/es/TextoDosColumnas/1371241563580/Trabajo_de_Fin_de_Grad}}
  \item[]
  \item Reglamento TFG de la FCSJ
  \item[] {\footnotesize \url{https://www.uc3m.es/secretaria-virtual/media/secretaria-virtual/doc/archivo/doc_reglamento-tfg-20-21/reglamento-tfg_sept_2020.pdf}}
\end{itemize}

\end{frame}
% ----------------------------------------------------

% ----------------------------------------------------
\begin{frame}
\frametitle{Personas}
\centering

\begin{itemize}
  \item \textbf{Tutor} de cada TFG
  \item \textbf{Coordinador}: se ocupa de las tareas académicas (organización ofertas y temas, tribunales, etc), no administrativas
  \item[]
  \item \textbf{Oficina de Estudiantes de Grado} (OEG) se ocupa de las \textbf{tareas administrativas} (matriculación, requisitos académicos, etc)
  \begin{itemize}
    \item Edificio Decanato
    \item \textbf{Rosa Blanca Martín} (\url{rosablanca.martin@uc3m.es})
    \item \textbf{Raúl Blanco} (\url{raul.blanco@uc3m.es}) (Director)
  \end{itemize}
\end{itemize}

\end{frame}
% ----------------------------------------------------


% ----------------------------------------------------
\begin{frame}
\frametitle{Normas básicas}
\centering


\begin{itemize}
  \item Condición para la evaluación de un TFG: Que el autor del trabajo no tenga pendientes de superar para la obtención del título más de 30 créditos ECTS, incluidos en ese cómputo los créditos de prácticas externas. (Y que sea depositado en la herramienta online.)
  \item Los estudiantes que, habiendo matriculado la asignatura, no cumplieran con el mínimo de créditos exigidos para su evaluación, podrán dispensar la convocatoria. Si el trabajo ya hubiese sido depositado e informado favorablemente por el tutor, el estudiante podrá presentar el mismo trabajo en la siguiente convocatoria a la que se presente.
  \item[]
  \item La OEG comprueba estos requisitos cuando se solicita la evaluación.
\end{itemize}


\end{frame}
% ----------------------------------------------------

% ----------------------------------------------------
\begin{frame}
\frametitle{Convocatorias y llamamientos}
\centering

\begin{itemize}
  \item Dos convocatorias: anticipada (otoño) y general (primavera)
  \item Cada convocatoria tiene \textbf{dos} llamamientos para la evaluación:
  \begin{itemize}
    \item Anticipada: Febrero y Julio/Septiembre
    \item General: Junio y Julio/Septiembre
  \end{itemize}
  \item Pero \textbf{sólo se puede elegir un llamamiento}
  \item Si no se va a ninguno: No presentado / Dispensa
  \item[] (Info sobre dispensa en la web de la OEG)
\end{itemize}

\end{frame}
% ----------------------------------------------------

% ----------------------------------------------------
\begin{frame}
\frametitle{Calendario otoño (convocatoria anticipada)}
\centering

\begin{itemize}
  \item Ofertas y asignación: septiembre
  \item Periodo tutorización: septiembre-diciembre
  \item[]
  \item \textbf{1er llamamiento:}
  \item[-] Entrega: finales enero/principios febrero
  \item[-] Defensa: febrero
  \item[]
  \item \textbf{2do llamamiento:}
  \item[-] Entrega: julio
  \item[-] Defensa: septiembre
  \item[] (La tutorización termina con el periodo lectivo del 1C)
\end{itemize}

\end{frame}
% ----------------------------------------------------

% ----------------------------------------------------
\begin{frame}
\frametitle{Calendario primavera (convocatoria general)}
\centering

\begin{itemize}
  \item Ofertas y asignación: inicio febrero
  \item Periodo tutorización: febrero-mayo
  \item[]
  \item \textbf{1er llamamiento:}
  \item[-] Entrega: primera mitad de junio
  \item[-] Defensa: finales junio / principios julio
  \item[]
  \item \textbf{2do llamamiento:}
  \item[-] Entrega: julio
  \item[-] Defensa: septiembre
  \item[] (La tutorización termina con el periodo lectivo del 2C)
\end{itemize}

\end{frame}
% ----------------------------------------------------

% ----------------------------------------------------
\begin{frame}
\frametitle{Evaluación}
\centering

\begin{itemize}
  \item \textbf{30\%} nota de seguimiento del \textbf{tutor}
  \begin{itemize}
    \item Media de la nota de 1) planificación, 2) seguimiento y 3) presentación
    \item También incorpora algo de la calidad del contenido del TFG
  \end{itemize}
  \item[]
  \item \textbf{70\%} nota del \textbf{tribunal} en defensa pública
  % \begin{itemize}
  %   \item Evalúa calidad académica y presentación
  % \end{itemize}
\end{itemize}

\end{frame}
% ----------------------------------------------------

% ----------------------------------------------------
\begin{frame}
\frametitle{Calificación tutor}
\centering

\begin{itemize}
  \item \textbf{Planificación y progreso de la tarea}
  \item[] {\small El estudiante ha asistido a las tutorías y actividades programadas y ha cumplido con los plazos indicados por el tutor. El estudiante ha desempeñado su labor con aprovechamiento.}
  \item \textbf{Seguimiento}
  \item[] {\small El estudiante ha seguido eficazmente las recomendaciones del tutor a la vez que ha mostrado iniciativa para buscar soluciones válidas y justificadas de forma autónoma.}
  \item \textbf{Presentación}
  \item[] {\small La memoria cumple los requisitos formales y de calidad exigidos.}
  \item[]
  \item \BGyellow{\textbf{Importante}} empezar temprano y tener contacto con el tutor (se puede suspender esta nota si estudiante trabaja al margen del tutor)
\end{itemize}

\end{frame}
% ----------------------------------------------------

% ----------------------------------------------------
\begin{frame}
\frametitle{Revisión de la nota del tutor}
\centering

\begin{itemize}
  \item El/la tutor/a \textbf{ha de comunicar por escrito} su nota (incluyendo los 3 componentes)
  \item El/la estudiante tiene derecho a solicitar \textbf{revisión}:
  \begin{itemize}
    \item Tutor/a fija lugar y fecha para revisión, \textbf{que ha de realizarse antes de que comiencen los tribunales}
  \end{itemize}
\end{itemize}

\end{frame}
% ----------------------------------------------------

% ----------------------------------------------------
\begin{frame}
\frametitle{Defensa pública}
\centering

\begin{itemize}
  \item Defensa pública de 20 minutos, que ha de ser \textbf{presencial}, salvo causas muy justicadas
  \begin{itemize}
    \item \textbf{10 minutos} de presentación oral
    \item \textbf{5 minutos} de preguntas del tribunal
    \item \textbf{5 minutos} para cambios entre estudiantes
  \end{itemize}
  \item Nota del tribunal:
  \begin{itemize}
    \item \textbf{Memoria}: 50\%
    \item \textbf{Metodología}: 30\%
    \item \textbf{Defensa oral}: 20\%
  \end{itemize}
  \item Al terminal, el tribunal comunicará su nota y la nota final a el/la estudiante, que firmará el acta si está conforme
  \item El/la estudiante puede solicitar revisión en ese mismo momento
\end{itemize}

\end{frame}
% ----------------------------------------------------

% ----------------------------------------------------
\begin{frame}
\frametitle{Nota del tribunal}
\centering

\begin{itemize}
  \item \textbf{Memoria (5):} contenido original, buen objeto de estudio y RQ, buena síntesis y análisis de la información, buena estructura y bien escrito
  \item \textbf{Metodología (3):} métodos coherentes con lo que se estudia y buena implementación
  \item \textbf{Defensa (2):} buena presentación oral, el/la estudiante ha sabido responder con soltura a las preguntas
\end{itemize}

\end{frame}
% ----------------------------------------------------

% ----------------------------------------------------
\begin{frame}
\frametitle{Revisión de la nota del tribunal}
\centering

\begin{itemize}
  \item El tribunal evaluador debe detallar la calificación de cada parte
  \item El/la estudiante dispone de un tiempo máximo de 5 minutos
  para exponer sus alegaciones por escrito, tras lo cual el tribunal formulará las
  explicaciones que considere convenientes y comunicará si la
  reclamación ha sido estimada
  \item El tribunal evaluador debe indicar en el acta si se ha solicitado la revisión y si esta ha supuesto algun cambio
  \item Tras esto, ya no hay revisiones de la nota
\end{itemize}

\end{frame}
% ----------------------------------------------------

% ----------------------------------------------------
\begin{frame}
\frametitle{Artículo 24: Revisión ante Decanato}
\centering

\begin{itemize}
  \item Un/a estudiante que considere que han producido irregularidades manifiestas en la evaluación de su TFG puede presentar un recurso ante la Decana en los 10 días siguientes a la fecha de su defensa
  \begin{itemize}
    \item El recurso debe presentarse por la sede electrónica de la FCSJ
  \end{itemize}
  \item \BGyellow{\textbf{No supone una revisión adicional para recalificar el TFG}}, la nota no se puede cambiar
  \item Solamente se admitirá a trámite si se han producido alguna de las siguientes situaciones:
  \begin{itemize}
    \item No realización del procedimiento de revisión
    \item Irregularidades manifiestas en la calificación del trabajo
  \end{itemize}
  \item En ausencia de dichas situaciones, el recurso puede desestimarse
\end{itemize}

\end{frame}
% ----------------------------------------------------

% ----------------------------------------------------
\begin{frame}
\frametitle{Artículo 24: Revisión ante Decanato}
\centering

\begin{itemize}
  \item El/la estudiante debe acreditar \textbf{por escrito} que se han producido dichas irregularidades
  \item La disconformidad con la calificación, en ausencia de dichas irregularidades, \BGyellow{no es motivo de recurso}
  \item Si el recurso se admite a trámite, la Decana convocará una Comisión que deliberará sobre la reclamación presentada
\end{itemize}

\end{frame}
% ----------------------------------------------------

% ----------------------------------------------------
\begin{frame}
\frametitle{Cierre de actas}
\centering

\begin{itemize}
  \item El tribunal envía las actas directamente a la OEG
  \item Tras unos días después del final de todos los tribunales, se cierran las actas y ya se puede solicitar el título
  \item Para dudas sobre esto: OEG
\end{itemize}

\end{frame}
% ----------------------------------------------------

% ----------------------------------------------------
\begin{frame}
\frametitle{Plagio}
\centering

\begin{itemize}
  \item Todos los trabajos pasan por \textbf{Turnitin}
  \item ``Los trabajos en los que se haya detectado plagio se calificarán con un cero-suspenso, haciéndose constar en el acta dicha circunstancia, sin perjuicio de la apertura del procedimiento disciplinario que, en su caso, proceda''
  \item \textbf{Turnitin también detecta el \BGyellow{uso de IA}}
\end{itemize}


\end{frame}
% ----------------------------------------------------

% ----------------------------------------------------
\begin{frame}
\frametitle{Tutorización}
\centering

\begin{itemize}
  \item Periodo de tutorización: desde la asignación \textbf{hasta el final del periodo lectivo}
  \begin{itemize}
    \item \textbf{Nota:} incluso aunque se posponga al segundo llamamiento
  \end{itemize}
  \item Tutorías (mínimo de 5h en total, al menos 3h individuales):
  \begin{itemize}
    \item Colectivas
    \item Individuales
  \end{itemize}
  \item Obligaciones del tutor además de las tutorías: escribir el informe tutor y \textbf{comunicar} la nota al estudiante \textbf{por escrito}
  \item[]
  \item De nuevo: Es \textbf{{\color{red}{importante}}} empezar a trabajar temprano y en contacto con el tutor
\end{itemize}

\end{frame}
% ----------------------------------------------------

% ----------------------------------------------------
\begin{frame}
\frametitle{Normas básicas del TFG}
\centering

\begin{itemize}
  \item Ver: \BGyellow{\textbf{GUÍA PARA ESTUDIANTES - TFG CCPP / EEII}}
  \item[] (La podéis encontrar en \url{https://ccss-uc3m.github.io/TFG/})
\end{itemize}

\end{frame}
% ----------------------------------------------------

% ----------------------------------------------------
\begin{frame}
\frametitle{Normas básicas del TFG}
\centering

\begin{itemize}
  \item \textbf{Idioma:} depende de la titulación (CP: español, EI: inglés)
  \item[]
  \item \textbf{Extensión:} 10.000 palabras, incluyendo \textbf{inclyendo todo} (también bibliografía), salvo Anexos
  \item \textbf{Tablas y gráficos:} en el texto principal, numeradas/os
  \item \textbf{Anexos/appendix:} al final, incluye toda la información adicional
  \item \textbf{Bibligrafía:} no hay un estilo prefijado, pero sí que ha de ser coherente y único: referencias ordenadas alfabéticamente al final, citas correctas: e.g. Smith (2010), etc
  \item \textbf{Portada:} estructura estándar con todos los datos necesarios (nombre, apellidos, NIU, email, título, nombre tutor...)
  \begin{itemize}
    \item Modelo disponible en la web
  \end{itemize}
\end{itemize}

\end{frame}
% ----------------------------------------------------

% ----------------------------------------------------
\begin{frame}
\frametitle{Normas básicas del TFG}
\centering

\begin{itemize}
  \item \textbf{Trabajo de investigación original}
  \item Pregunta de investigación, y evidencia empírica para responderla
  \begin{itemize}
    \item Valen tanto datos \textbf{cualitativos} como \textbf{cuantitativos}
    \item \textbf{Trabajos teóricos}: contenido original
  \end{itemize}
  \item Estructura típica
  \begin{itemize}
    \item (Título y abstract/resumen)
    \item Introducción
    \item Revisión de la literature
    \item Marco teórico
    \item Resultados y discusión
    \item Conclusiones
    \item (Bibliografía y appendix)
  \end{itemize}
\end{itemize}

\end{frame}
% ----------------------------------------------------

% ----------------------------------------------------
\begin{frame}
\frametitle{}
\centering

Preguntas?

\end{frame}
% ----------------------------------------------------

% ====================================================
\end{document}
